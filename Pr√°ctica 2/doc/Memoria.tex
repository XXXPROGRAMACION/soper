\documentclass[]{article}

\title{Memoria de la práctica 2\\
	\Large Sistemas operativos 2018-2019}

\author{Alejandro Pascual y V\'ictor Yrazusta}
\usepackage{amssymb}
\usepackage{graphicx}
\usepackage[spanish, es-tabla]{babel}
\usepackage[legalpaper, margin=1in]{geometry}

\begin{document}

\maketitle

\section*{Respuestas a las preguntas cortas}
\subsection*{Ejercicio 2}
\subsubsection*{¿Qué mensajes imprime cada hijo? ¿Por qué?}
Cada hijo imprime únicamente el mensaje previo al sleep, ya que su padre les manda la señal SIGTERM antes de que se complete su espera.

\subsection*{Ejercicio 3}
\subsubsection*{a) ¿La llamada a signaction supone que se ejecute la función manejador?}
No, la llamada a sigaction sustituye el manejador predeterminado por el definido en el programa, pero este no será ejecutado hasta que no llegue la señal SIGINT.

\subsubsection*{b) ¿Cuándo aparece el printf en pantalla?}
El printf en main se imprime al inicio y cada 9999 segundos. El printf en el manejador se imprime cuando llega la señal SIGINT.

\subsubsection*{c) ¿Qué ocurre por defecto cuando un programa recibe una señal y no la tiene capturada?}
Si no se ha definido un manejador propio se ejecutará el predeterminado, que interrumpe la ejecución del proceso.

\subsubsection*{d) Modifica el programa anterior en un nuevo ejercicio3d.c que capture la señal SIGKILL ycuya función manejadora escriba  ``He conseguido capturar SIGKILL''. ¿Por qué nunca sale por pantalla ``He conseguido capturar SIGKILL''?}
El manejador nunca se ejecutaría, ya que la señal SIGKILL nunca llega al proceso, sino que es manejada por el sistema operativo. Además, en tiempo de ejecución recibimos un error de sigaction por argumento inválido.

\subsection*{Ejercicio 6}
\subsubsection*{¿Qué sucede cuando el hijo recibe la señal de alarma?}
Cuando el hijo recibe la alarma, pueden pasar dos cosas. Si está ``contando'' tendrá una máscara para la señal SIGALRM, por lo que la ignorará hasta terminar de contar. Si no está ``contando'' terminará inmediatamente, independientemente de si está bloqueado por sleep() o no.

\subsection*{Ejercicio 8}
\subsubsection*{a) ¿Qué pasa cuando SECS=0 y N\_READ=1? ¿Se producen lecturas? ¿Se producen escrituras? ¿Por qué?}
Se producen lecturas y escrituras sucesivamente, una a una ya que solo hay un escritor y un lector. Esto se debe a que el lector cambia el semáforo de escritura tras cada lectura, permitiendo que se intercalen sus ejecuciones.

\subsubsection*{b) ¿Qué pasa cuando SECS=1 y N\_READ=10? ¿Se producen lecturas? ¿Se producen escrituras? ¿Por qué?}
Se produce una escritura por cada 10 lecturas. Esto se debe a que los tiempos de ejecución son mucho menores que los tiempos de espera, por lo que los lectores modifican el semáforo de escritura una vez se han ejecutado los 10.

\subsubsection*{c) ¿Qué pasa cuando SECS=0 y N\_READ=10? ¿Se producen lecturas? ¿Se producen escrituras? ¿Por qué?}
Principalmente se producen lecturas y, aunque en la práctica podemos observar escrituras, es posible que el escritor no pueda ejecutarse nunca. Esto se debe a que los lectores pueden volver a la cola de lecturas nada más terminar de leer, por lo que es posible que la cola nunca se vacíe, impidiendo la escritura.

\subsubsection*{d) ¿Qué pasa si los procesos escritores/lectores no duermen nada entre escrituras/lecturas (si se elimina totalmente el sleep del bucle)? ¿Se producen lecturas? ¿Se producen escrituras? ¿Por qué?}
En este caso la respuesta teórica es muy similar a la anterior, sin embargo, en la práctica, en cuanto es el turno de  los lectores el escritor no vuelve a ejecutarse. \\

Esta diferencia puede deberse al funcionamiento de la función sleep(), sospechamos que aunque el tiempo indicado sea 0, el programa es interrumpido de todos modos. Esto supondría aumentar mucho la probabilidad de que, en el caso anterior, al escritor le toque ejecutarse justo cuando ningún lector está leyendo. En este caso, sin embargo, que se de dicha situación es tan poco probable que no hemos podido llegar a observarla.

\section*{Detalles de implementación}
\subsection*{Ejercicio 2}
Hemos escrito el código siguiendo el enunciado y sin mayores modificaciones o complicaciones. Utilizamos un bucle en el que el padre crea un bucle, duerme 5 segundos y hace que el hijo termine.

\subsection*{Ejercicio 3}
Hemos modificado el programa de referencia sustituyendo SIGTERM por SIGKILL. Cabe destacar que el propio entorno de desarrollo nos avisaba de que cometíamos un error al tratar de capturar SIGKILL.

\subsection*{Ejercicio 4}
Hemos implementado el programa tal y como venía definido en el enunciado. El padre crea un gestor, que a su vez crea N hijos (que le van confirmando su correcta preparación), y este le avisa cuando todos los participantes están listos. Cuando esto ocurre el padre da la señal de salida y el resultado varía en función de la administración de la multiprogramación por parte del procesador. \\

Hemos tenido que darles un manejador sin funcionalidad tanto al padre como al gestor. Esto se debe a que cuando kill manda una señal a un grupo de procesos se la manda a todos, también a quién ha ejecutado kill. Como no queríamos que se imprimiese en pantalla el feedback por defecto para una señal sin manejador propio, les hemos dado un manejador vacío. 

\subsection*{Ejercicio 6}
a) Hemos seguido las instrucciones del ejercicio sin complicaciones. Para establecer qué señales ignorar utilizamos sigprocmask(SIG\_BLOCK, ...) y para dejar de bloquearlas sigprocmask(SIG\_UNBLOCK, ...). \\

b) Cabe destacar que en esta versión, al no tener control sobre cuando puede ser interrumpido el programa, es posible que el printf() utilzado para contar sea interrumpido a mitad de ejecución, por lo que usamos fflush() para evitar problemas al imprimir el mensaje final.

\subsection*{Ejercicio 8}
Hemos implementado el ejercicio aplicando el algoritmo especificado. Tal y como queda reflejado en las respuestas a las preguntas anteriores, este algoritmo da prioridad a la ejecución continua de lectores. Esto supone que en determinadas condiciones (como la existencia de un gran número de lectores o un tiempo de espera muy corto) el escritor pueda sufrir inanición.

\subsection*{Ejercicio 9}
Hemos creado tres semáforos para gestionar la concurrencia del padre y los hijos. Uno de ellos (sem\_hijos) nos permite hacer que solo haya un hijo esperando a escribir en el fichero, para que el padre solo tenga que esperar a un hijo como máximo. El segundo (sem\_fichero) controla la lectura y escritura en el fichero. El tercero (sem\_carrera) nos permite asegurarnos de que todos los procesos comienzan la carrera a la vez.

\end{document}